
\documentclass[a4paper,12pt]{paper}
\usepackage{amsmath}
\usepackage{amssymb}
\usepackage{titlesec}
\usepackage{fullpage}
\usepackage{hyperref}
\usepackage[framemethod=TikZ]{mdframed}
\usepackage[a4paper,bindingoffset=0.2in,%
            left=.8in,right=.8in,top=.5in,bottom=1in,%
            footskip=.25in]{geometry}
\usepackage[outputdir=out]{minted}
\usepackage{xepersian}
\settextfont{Roya}
\deflatinfont\dast{Far.Dast Nevis}
\deflatinfont\en{Ubuntu}
\deflatinfont\code{Ubuntu Mono}
\newfontfamily\monaco{Ubuntu Mono}[NFSSFamily=Ubuntu]
\definecolor{codebg}{rgb}{0.1,0.1, 0.15}
\usemintedstyle{monokai}
\setminted[java]{
				fontfamily=Ubuntu,
				linenos,
				bgcolor=codebg
			}
\newcommand{\minti}[1]{\lr{\mintinline{java}{#1}}}
%\newenvironment{sugg}{\begin{mdframed} {\dast پیشنهاد: }}{\end{mdframed}}

\newmdenv[%
outerlinewidth=2,%
roundcorner=5pt,%
leftmargin=40,%
rightmargin=40,%
outerlinecolor=blue!50,%
frametitleaboveskip=-\ht\strutbox,
frametitle= \colorbox{white}{\space{\dast پیشنهاد}\space}%
] {sugg}

\titleformat{\section}[runin]{\normalfont\bfseries}{\thesection}{0.7em}{}
\titlespacing{\section}{.5pc}{2ex plus .1ex minus .2ex}{1pc}
\titleformat{\subsection}[runin]{\normalfont\bfseries}{\thesubsection}{0.7em}{}
\titlespacing{\subsection}{.5pc}{2ex plus .1ex minus .2ex}{1pc}

\begin{document}
\noindent{
\sc{ویرایش \lr{0.1}} \\
\large{\textbf{پروژه‌ی درسِ برنامه‌سازیِ پیشرفته} \\
 \emph{\en{UTank}}
}
}
\\

\section*{مقدمه}
در این مستند، همراه با معرفیِ پروژه‌ی درس، با ابزارهای گرافیکیِ {\en Java} نیز آشنا می‌شوید. پروژه‌ی درس یک بازیِ گرافیکیِ ساده است که توسطِ دو بازیکن و با صفحه‌کلید بازی می‌شود. هدف از انجامِ پروژه، علاوه بر سنجشِ مهارتِ طراحی و پیاده‌سازیِ نرم‌افزار، تمرینِ کارِ گروهی و استفاده‌  از ابزارهای شیء‌گراییِ {\en Java} در یک مسئله‌ی دنیایِ واقعی است. بنابراین، در معیارِ نهاییِ سنجشِ پروژه، هم کارآییِ برنامه و هم کدِ آن مؤثر خواهند بود.

همچنین پیشنهادهایی برای پیاده‌سازیِ پروژه داده شده است. این پیشنهادها می‌توانند مستقیماً و یا تنها به عنوانِ طرحی کلی استفاده شوند، و همچنین می‌توانند کاملاً نادیده گرفته شوند.

توصیه می‌شود پیش از شروع به کار، این مستند را به دقت و تا انتها خوانده و سؤالات یا ابهام‌های احتمالی را برطرف کنید.

\section{\en{Swing}}
پیش‌تر با چگونگیِ ارتباط با کاربر از طریقِ رابطِ کاربریِ {\en command-line} آشنا شده‌ایم. حالا می‌خواهیم از محیطِ گرافیکی برای این کار استفاده کنیم. رابطِ کاربریِ گرافیکی به ما اجازه می‌دهد تا به راحتی تصویری دلخواه برای نمایش به کاربر بسازیم و همچنین ورودی‌های کاربر مانندِ صفحه‌کلید و موس را کنترل کنیم. در واقع رابطِ گرافیکی برای ما یک «پنجره» با صفحه‌ای خالی فراهم خواهد کرد که می‌توانیم روی آن «رسم» کنیم و بر اساس تعاملِ کاربر با این پنجره کارکردِ برنامه را تعیین کنیم.

ابزارهایی که برای ساختِ یک محیطِ گرافیکی در {\en Java} استفاده خواهیم کرد در پکیجی به نامِ {\en Swing} در {\en JDK} حضور دارند. اولین ابزاری که معرفی خواهد شد، کلاسی به نامِ \minti{JFrame} است که هر {\en object}ِ آن، یک پنجره‌ی رابطِ گرافیکی است.

\subsection{\en JFrame}
اجرای قطعه‌کدِ زیر یک پنجره‌ی خالی به ابعادِ ۵۰۰ در ۵۰۰ پیکسل را نمایش می‌دهد.

\begin{minted}{java}
import javax.swing.*;

public class Main {
  public static void main(String[] args) {
    JFrame jframe = new JFrame();
    jframe.setSize(500, 500);
    jframe.setVisible(true);
  }
}
\end{minted}
تابع‌های \minti{.setSize()} و \minti{.setVisible()} دو متد از کلاسِ {\en JFrame} هستند که به ترتیب ابعادِ پنجره را مقدار دهی کرده و آن را نمایش می‌دهند. متدهای دیگری از {\en JFrame} نیز در بخش‌های بعدی معرفی خواهند شد.

\begin{sugg}
  در واقع \minti{JFrame} یکی از راه‌های ساختِ پنجره در {\en Swing} است. می‌توانید از هر یک از زیرکلاس‌های کلاسِ \minti{Window} که \minti{JFrame} یکی از آن‌ها است به همین صورت استفاده کنید.
\end{sugg}

{\en Swing}
شاملِ تعدادی کلاسِ {\en component} است که اجزای مختلفِ یک پنجره را می‌سازند. {\en component}های مختلفی مانندِ دکمه، لیست یا عکس قابلِ استفاده هستند. یک پنجره می‌تواند شامل تعدادی {\en component} باشد. در بخش‌های بعدی با یک {\en component}ِ ساده‌ی {\en Swing} به نامِ \minti{JPanel} آشنا شده و نحوه‌ی اضافه کردنِ آن به پنجره و رسمِ شکل بر روی آن را می‌آموزیم.
\subsection{\en JPanel}
یک 

\end{document}

